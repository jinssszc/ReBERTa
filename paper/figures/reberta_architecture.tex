\documentclass{standalone}
\usepackage{tikz}
\usetikzlibrary{shapes,arrows,positioning,fit,backgrounds}

\begin{document}
\begin{tikzpicture}[
    node distance=0.7cm and 1cm,
    box/.style={rectangle, draw, rounded corners, fill=blue!20, minimum width=3cm, minimum height=1cm, text centered},
    arrow/.style={thick,->,>=stealth},
    largetext/.style={font=\large},
    mediumtext/.style={font=\normalsize},
    smalltext/.style={font=\small},
    process/.style={rectangle, draw, fill=green!20, rounded corners, minimum width=3cm, minimum height=1cm, text centered}
]

% 输入层
\node[box] (input) {长文本输入};

% 窗口划分
\node[process, below=of input] (windowing) {窗口划分 \& 预处理};

% 窗口处理部分
\node[box, below=of windowing] (window1) {窗口 1};
\node[box, right=of window1] (window2) {窗口 2};
\node[box, right=of window2] (window3) {窗口 ... N};

% 窗口处理层标题
\node[largetext, above=0.1cm of window1, xshift=2cm] {窗口处理层};
\draw[dashed] ([xshift=-0.5cm, yshift=0.5cm]window1.north west) rectangle ([xshift=0.5cm, yshift=-0.5cm]window3.south east);

% 窗口间信息传递
\node[process, below=1.5cm of window2] (info_transfer) {窗口间门控信息传递};

% 全局聚合层
\node[box, below=of info_transfer] (global_aggregation) {全局聚合层};

% 最终分类
\node[box, below=of global_aggregation] (output) {分类输出};

% 迭代回路
\draw[arrow, dashed, thick, red] (global_aggregation) -- ++(3cm,0) |- ([yshift=0.3cm]window3.east);
\node[smalltext, xshift=3.5cm, yshift=1.5cm] at (global_aggregation) {多轮迭代};

% 连接
\draw[arrow] (input) -- (windowing);
\draw[arrow] (windowing) -- (window1);
\draw[arrow] (windowing) -- (window2);
\draw[arrow] (windowing) -- (window3);
\draw[arrow] (window1) -- ([xshift=-1cm]info_transfer.north);
\draw[arrow] (window2) -- (info_transfer);
\draw[arrow] (window3) -- ([xshift=1cm]info_transfer.north);
\draw[arrow] (info_transfer) -- (global_aggregation);
\draw[arrow] (global_aggregation) -- (output);

% RoPE标注
\node[smalltext, rotate=90] at ([xshift=-0.5cm]window1.west) {旋转位置编码 (RoPE)};

\end{tikzpicture}
\end{document}
